\documentclass[a4paper,11pt]{article}
\usepackage{fullpage}
\usepackage[utf8x]{inputenc}
\usepackage{amsmath}
\usepackage{amsthm}
\usepackage{amssymb}
\usepackage{graphicx}
\usepackage[hypcap=false]{caption}
\usepackage{subcaption}
\usepackage{chemfig}

% Title Page
\title{a}
\author{}

\begin{document}
\maketitle

Example taken from the answer at http://tex.stackexchange.com/questions/56275/align-subfigures-vertically.

\begin{figure}[ht] \centering
        \begin{subfigure}{\textwidth}
                \centering
                \chemfig{C(=[:0]O)(=[:180]O)}\;
                \chemfig{H-[::37.775,2]O-[::-75.55,2]H}\;
                \caption{}
                \label{fig:database}
        \end{subfigure}\\[2em]
        %~ %add desired spacing between images, e. g. ~, \quad, \qquad etc. 
          %(or a blank line to force the subfigure onto a new line)
        \begin{subfigure}[b]{\textwidth}
                \centering
                \chemfig{C(-[:90]H)(-[:270]H)(=[:0]C(=[:0]C(-[:90]H)(-[:270]H)))}
                \caption{}
                \label{fig:partb}
        \end{subfigure}
        \caption{}\label{fig:query}
\end{figure}

In Fig. \ref{fig:query} there are two subfigures, \ref{fig:database} and \ref{fig:partb}

\end{document}