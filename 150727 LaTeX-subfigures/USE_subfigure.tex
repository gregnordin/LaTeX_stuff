\documentclass[12pt]{article}
\usepackage[utf8]{inputenc}
\usepackage[letterpaper,top=2.5cm,bottom=2.5cm,left=2cm,right=2cm]{geometry}
\usepackage{amsmath}
%\usepackage{upgreek}
\usepackage{enumerate}
\usepackage{indentfirst}
\usepackage{subfigure}
\usepackage{graphicx}
\usepackage{caption}
\usepackage{siunitx}
\usepackage{array}
\usepackage{multirow}
\usepackage{chemfig}

\setlength{\parindent}{4ex}

\title{Test subfigure module to make subfigures}
\author{G Nordin}
\date{\today}

\begin{document}
\maketitle

\noindent Let's include two figures.

\begin{figure}[ht] \centering
        \begin{subfigure}{\textwidth}
                \centering
                \chemfig{C(=[:0]O)(=[:180]O)}\;
                \chemfig{H-[::37.775,2]O-[::-75.55,2]H}\;
                \caption{}
                \label{fig:database}
        \end{subfigure}\\[2em]
        %~ %add desired spacing between images, e. g. ~, \quad, \qquad etc. 
          %(or a blank line to force the subfigure onto a new line)
        \begin{subfigure}[b]{\textwidth}
                \centering
                \chemfig{C(-[:90]H)(-[:270]H)(=[:0]C(=[:0]C(-[:90]H)(-[:270]H)))}
                \caption{}
                \label{fig:partb}
        \end{subfigure}
        \caption{}\label{fig:query}
\end{figure}

In Fig. \ref{fig:query} there are two subfigures, \ref{fig:database} and \ref{fig:partb}

some text. some more text. \ref{fgr:beerslaw1}

\noindent And now we'll reference them as Fig. \ref{fgr:beerslaw1} and Fig. \ref{fgr:beerslaw2}.


\end{document}